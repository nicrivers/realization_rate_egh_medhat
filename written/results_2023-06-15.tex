\documentclass[8pt]{beamer}

\usepackage{threeparttable}
\usepackage{array}
\usepackage{graphicx}
\usepackage{tabularx}
\usepackage{booktabs}




\begin{document}

\begin{frame}{Codebook}
	Unless otherwise stated, column definitions are as follows
	\begin{enumerate}
		\item Treated + all available controls
		\item Treated + 1:1 matched controls based on pre-treatment consumption
		\item Treated + 1:1 matched controls based on building characteristics
		\item Treated + 1:1 matched controls based on pre-treatment consumption and building characteristics
		\item Treated only
	\end{enumerate}
\end{frame}

\begin{frame}{TWFE estimates for total energy; monthly data}
	
\begingroup
\centering
\begin{tabular}{lccccc}
   \tabularnewline \midrule \midrule
   Dependent Variable: & \multicolumn{5}{c}{log(energy)}\\
   Model:             & (1)             & (2)             & (3)             & (4)             & (5)\\  
   \midrule
   \emph{Variables}\\
   treated\_postTRUE  & -0.1571$^{***}$ & -0.1453$^{***}$ & -0.1512$^{***}$ & -0.1396$^{***}$ & -0.1573$^{***}$\\   
                      & (0.0063)        & (0.0072)        & (0.0070)        & (0.0074)        & (0.0107)\\   
   \midrule
   \emph{Fixed-effects}\\
   id                 & Yes             & Yes             & Yes             & Yes             & Yes\\  
   cons\_date         & Yes             & Yes             & Yes             & Yes             & Yes\\  
   \midrule
   \emph{Fit statistics}\\
   Observations       & 2,882,662       & 429,578         & 429,833         & 429,866         & 75,461\\  
   R$^2$              & 0.79219         & 0.80557         & 0.81004         & 0.81181         & 0.81352\\  
   Within R$^2$       & 0.00267         & 0.01011         & 0.01134         & 0.00962         & 0.01377\\  
   \midrule \midrule
   \multicolumn{6}{l}{\emph{Clustered (id \& cons\_date) standard-errors in parentheses}}\\
   \multicolumn{6}{l}{\emph{Signif. Codes: ***: 0.01, **: 0.05, *: 0.1}}\\
\end{tabular}
\par\endgroup



\end{frame}

\begin{frame}{TWFE estimates for gas; monthly data}
	
\begingroup
\centering
\begin{tabular}{lccccc}
   \tabularnewline \midrule \midrule
   Dependent Variable: & \multicolumn{5}{c}{log(gas)}\\
   Model:             & (1)             & (2)             & (3)             & (4)             & (5)\\  
   \midrule
   \emph{Variables}\\
   treated\_postTRUE  & -0.2081$^{***}$ & -0.1944$^{***}$ & -0.1993$^{***}$ & -0.1901$^{***}$ & -0.2196$^{***}$\\   
                      & (0.0062)        & (0.0076)        & (0.0074)        & (0.0078)        & (0.0089)\\   
   \midrule
   \emph{Fixed-effects}\\
   id                 & Yes             & Yes             & Yes             & Yes             & Yes\\  
   cons\_date         & Yes             & Yes             & Yes             & Yes             & Yes\\  
   \midrule
   \emph{Fit statistics}\\
   Observations       & 2,874,433       & 428,688         & 429,313         & 429,103         & 75,699\\  
   R$^2$              & 0.85408         & 0.85549         & 0.85894         & 0.86033         & 0.84099\\  
   Within R$^2$       & 0.00384         & 0.01381         & 0.01480         & 0.01352         & 0.01773\\  
   \midrule \midrule
   \multicolumn{6}{l}{\emph{Clustered (id \& cons\_date) standard-errors in parentheses}}\\
   \multicolumn{6}{l}{\emph{Signif. Codes: ***: 0.01, **: 0.05, *: 0.1}}\\
\end{tabular}
\par\endgroup



\end{frame}

\begin{frame}{TWFE estimates for electricity; monthly data}
	
\begingroup
\centering
\begin{tabular}{lccccc}
   \tabularnewline \midrule \midrule
   Dependent Variable: & \multicolumn{5}{c}{log(elec)}\\
   Model:             & (1)                   & (2)                   & (3)                   & (4)                   & (5)\\  
   \midrule
   \emph{Variables}\\
   treated\_postTRUE  & -0.0387$^{***}$       & -0.0138               & -0.0084               & 0.0015                & 0.0207\\   
                      & (0.0099)              & (0.0121)              & (0.0123)              & (0.0117)              & (0.0155)\\   
   \midrule
   \emph{Fixed-effects}\\
   id                 & Yes                   & Yes                   & Yes                   & Yes                   & Yes\\  
   cons\_date         & Yes                   & Yes                   & Yes                   & Yes                   & Yes\\  
   \midrule
   \emph{Fit statistics}\\
   Observations       & 2,908,550             & 432,238               & 432,263               & 432,424               & 76,430\\  
   R$^2$              & 0.46709               & 0.50410               & 0.49463               & 0.49444               & 0.54453\\  
   Within R$^2$       & $8.72\times 10^{-5}$  & $5.66\times 10^{-5}$  & $2.07\times 10^{-5}$  & $7.29\times 10^{-7}$  & 0.00014\\  
   \midrule \midrule
   \multicolumn{6}{l}{\emph{Clustered (id \& cons\_date) standard-errors in parentheses}}\\
   \multicolumn{6}{l}{\emph{Signif. Codes: ***: 0.01, **: 0.05, *: 0.1}}\\
\end{tabular}
\par\endgroup



\end{frame}

\begin{frame}{Sun + Abraham estimates}
	The Sun and Abraham correction is estimated by interacting a cohort dummy with a time-to-treatment dummy. In our monthly data, we observe 49 different ``cohorts'' (i.e., households retrofit in 49 different months).  Our data has 250 different ``time to treatments'' (i.e., months before and after retrofit). This implies estimating a model with about 10,000 dummy variables.  This is not feasible. Instead, to estimate the Sun and Abraham model, I convert to annual data first. The results first show the TWFE estimates based on annual data.
\end{frame}

\begin{frame}{TWFE estimates for energy; annual data}
	
\begingroup
\centering
\begin{tabular}{lccccc}
   \tabularnewline \midrule \midrule
   Dependent Variable: & \multicolumn{5}{c}{log(energy)}\\
   Model:             & (1)             & (2)             & (3)             & (4)             & (5)\\  
   \midrule
   \emph{Variables}\\
   treated\_postTRUE  & -0.1682$^{***}$ & -0.1573$^{***}$ & -0.1605$^{***}$ & -0.1489$^{***}$ & -0.1870$^{***}$\\   
                      & (0.0060)        & (0.0079)        & (0.0077)        & (0.0090)        & (0.0114)\\   
   \midrule
   \emph{Fixed-effects}\\
   id                 & Yes             & Yes             & Yes             & Yes             & Yes\\  
   consyear           & Yes             & Yes             & Yes             & Yes             & Yes\\  
   \midrule
   \emph{Fit statistics}\\
   Observations       & 246,541         & 35,175          & 35,165          & 35,178          & 5,071\\  
   R$^2$              & 0.80826         & 0.84582         & 0.85472         & 0.85053         & 0.90145\\  
   Within R$^2$       & 0.01388         & 0.06664         & 0.07291         & 0.06048         & 0.11563\\  
   \midrule \midrule
   \multicolumn{6}{l}{\emph{Clustered (id \& consyear) standard-errors in parentheses}}\\
   \multicolumn{6}{l}{\emph{Signif. Codes: ***: 0.01, **: 0.05, *: 0.1}}\\
\end{tabular}
\par\endgroup



\end{frame}

\begin{frame}{Sun+Abraham estimates for energy; annual data}
	
\begingroup
\centering
\begin{tabular}{lccccc}
   \tabularnewline \midrule \midrule
   Dependent Variable: & \multicolumn{5}{c}{log(energy)}\\
   Model:       & (1)             & (2)             & (3)             & (4)             & (5)\\  
   \midrule
   \emph{Variables}\\
   ATT          & -0.1668$^{***}$ & -0.1544$^{***}$ & -0.1577$^{***}$ & -0.1468$^{***}$ & -0.1725$^{***}$\\   
                & (0.0037)        & (0.0049)        & (0.0047)        & (0.0048)        & (0.0155)\\   
   \midrule
   \emph{Fixed-effects}\\
   id           & Yes             & Yes             & Yes             & Yes             & Yes\\  
   consyear     & Yes             & Yes             & Yes             & Yes             & Yes\\  
   \midrule
   \emph{Fit statistics}\\
   Observations & 246,541         & 35,175          & 35,165          & 35,178          & 5,071\\  
   R$^2$        & 0.80831         & 0.84640         & 0.85521         & 0.85139         & 0.90159\\  
   Within R$^2$ & 0.01414         & 0.07016         & 0.07603         & 0.06588         & 0.11683\\  
   \midrule \midrule
   \multicolumn{6}{l}{\emph{Clustered (id \& consyear) standard-errors in parentheses}}\\
   \multicolumn{6}{l}{\emph{Signif. Codes: ***: 0.01, **: 0.05, *: 0.1}}\\
\end{tabular}
\par\endgroup



\end{frame}

\begin{frame}{Sun+Abraham estimates for gas; annual data}
	
\begingroup
\centering
\begin{tabular}{lccccc}
   \tabularnewline \midrule \midrule
   Dependent Variable: & \multicolumn{5}{c}{log(gas)}\\
   Model:       & (1)             & (2)             & (3)             & (4)             & (5)\\  
   \midrule
   \emph{Variables}\\
   ATT          & -0.1958$^{***}$ & -0.1848$^{***}$ & -0.1890$^{***}$ & -0.1781$^{***}$ & -0.2028$^{***}$\\   
                & (0.0042)        & (0.0055)        & (0.0053)        & (0.0054)        & (0.0165)\\   
   \midrule
   \emph{Fixed-effects}\\
   id           & Yes             & Yes             & Yes             & Yes             & Yes\\  
   consyear     & Yes             & Yes             & Yes             & Yes             & Yes\\  
   \midrule
   \emph{Fit statistics}\\
   Observations & 247,154         & 35,234          & 35,244          & 35,248          & 5,071\\  
   R$^2$        & 0.82162         & 0.84958         & 0.86034         & 0.85334         & 0.89767\\  
   Within R$^2$ & 0.01857         & 0.08732         & 0.09513         & 0.08423         & 0.14043\\  
   \midrule \midrule
   \multicolumn{6}{l}{\emph{Clustered (id \& consyear) standard-errors in parentheses}}\\
   \multicolumn{6}{l}{\emph{Signif. Codes: ***: 0.01, **: 0.05, *: 0.1}}\\
\end{tabular}
\par\endgroup



\end{frame}

\begin{frame}{Sun+Abraham estimates for electricity; annual data}
	
\begingroup
\centering
\begin{tabular}{lccccc}
   \tabularnewline \midrule \midrule
   Dependent Variable: & \multicolumn{5}{c}{log(elec)}\\
   Model:       & (1)             & (2)            & (3)           & (4)      & (5)\\  
   \midrule
   \emph{Variables}\\
   ATT          & -0.0495$^{***}$ & -0.0265$^{**}$ & -0.0195$^{*}$ & -0.0146  & 0.0068\\   
                & (0.0071)        & (0.0093)       & (0.0101)      & (0.0093) & (0.0310)\\   
   \midrule
   \emph{Fixed-effects}\\
   id           & Yes             & Yes            & Yes           & Yes      & Yes\\  
   consyear     & Yes             & Yes            & Yes           & Yes      & Yes\\  
   \midrule
   \emph{Fit statistics}\\
   Observations & 246,819         & 35,183         & 35,170        & 35,180   & 5,071\\  
   R$^2$        & 0.64327         & 0.70877        & 0.70024       & 0.70385  & 0.86097\\  
   Within R$^2$ & 0.00060         & 0.00172        & 0.00150       & 0.00112  & 0.00091\\  
   \midrule \midrule
   \multicolumn{6}{l}{\emph{Clustered (id \& consyear) standard-errors in parentheses}}\\
   \multicolumn{6}{l}{\emph{Signif. Codes: ***: 0.01, **: 0.05, *: 0.1}}\\
\end{tabular}
\par\endgroup



\end{frame}

\begin{frame}{Event study plot}
	\centering
	\includegraphics[width=0.75\linewidth]{../output_figures_tables/event_study.png}
\end{frame}

\begin{frame}{Manual cohort-by-cohort analysis}
	The Sun and Abraham estimator (like other new TWFE estimators) is simply as weighted aggregate of DID estimators for all treated cohorts. To better understand what's going on, I am going to manually estimate treatment effects for all cohorts.  In doing so, I can also decompose the treatment effect into three types of comparisons (a la Goodman-Bacon): treated vs. never treated; later treated vs. earlier treated; and earlier treated vs. later treated.  Using earlier treatments as controls is the potentially problematic case.
\end{frame}

\begin{frame}{Dynamic cohort effects}
	In the following slide, I use the participants only sample, and only make ``good'' comparisons -- i.e., comparing earlier treated (the treatment group) with not-yet-treated (the control group) households. I run the regression for the 2008, 2009, and 2010 cohorts.  I drop years between the pre- and post-retrofits.  The treatment effects are very similar to what we observed earlier under the TWFE estimator. They are NOT the same as under the Sun and Abraham estimator.
\end{frame}

\begin{frame}{Dynamic cohort effects}
	\input{../output_figures_tables/dynamic_cohort_estimates.tex}
\end{frame}

\begin{frame}{Now, I use the data WITH untreated households, and illustrate DID results from all possible comparisons.  I do this graphically, but also show a regression coefficient.  I didn't drop the between-retrofit period in producing these figures (although I could).}
\end{frame}

\begin{frame}{Treated vs. never treated}
	\includegraphics[width=\linewidth]{../output_figures_tables/cp2008}
\end{frame}

\begin{frame}{Treated vs. never treated}
	\includegraphics[width=\linewidth]{../output_figures_tables/cp2009}
\end{frame}

\begin{frame}{Treated vs. never treated}
	\includegraphics[width=\linewidth]{../output_figures_tables/cp2010}
\end{frame}

\begin{frame}{Treated vs. never treated}
	\includegraphics[width=\linewidth]{../output_figures_tables/cp2011}
\end{frame}

\begin{frame}{Treated vs. never treated}
	\includegraphics[width=\linewidth]{../output_figures_tables/cp2012}
\end{frame}

\begin{frame}{Earlier treated vs. later treated}
	\includegraphics[width=\linewidth]{../output_figures_tables/el2008}	
\end{frame}

\begin{frame}{Earlier treated vs. later treated}
	\includegraphics[width=\linewidth]{../output_figures_tables/el2009}	
\end{frame}

\begin{frame}{Earlier treated vs. later treated}
	\includegraphics[width=\linewidth]{../output_figures_tables/el2010}	
\end{frame}

\begin{frame}{Later treated vs. earlier treated}
	\includegraphics[width=\linewidth]{../output_figures_tables/le2009}	
\end{frame}

\begin{frame}{Later treated vs. earlier treated}
	\includegraphics[width=\linewidth]{../output_figures_tables/le2010}	
\end{frame}

\begin{frame}{Later treated vs. earlier treated}
	\includegraphics[width=\linewidth]{../output_figures_tables/le2011}	
\end{frame}

\begin{frame}{Later treated vs. earlier treated}
	\includegraphics[width=\linewidth]{../output_figures_tables/le2012}	
\end{frame}

\begin{frame}{Summary}
	All of these comparisons reveal a coefficient of around 16\%, just like our main estimates in the current version of the paper.  This leads me to think that I have not estimated the Sun and Abraham model properly, and that the effect of retrofits really is an energy saving of approximately 16\%. Moreover, the event study plot in slide 9 looks ``wrong'': with the particpants only sample, it should not be possible to estimate coefficients on treatment+4 years or more, since we don't observe an appropriate control group.
\end{frame}
\end{document}